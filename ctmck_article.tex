%=================================================================
%  CTMCK – ARTICLE A : MECHANICS & GAUGE ORIGIN (updated)
%=================================================================
\documentclass[aps,prd,reprint,amsmath,amssymb,nofootinbib]{revtex4-2}

%------------------------------------------------
%              PACKAGES
%------------------------------------------------
\usepackage{graphicx}
\usepackage{dcolumn}
\usepackage{bm}
\usepackage{hyperref}
\usepackage{physics}
\usepackage{mathtools}
\usepackage{tikz}
\usetikzlibrary{3d,arrows.meta,calc,decorations.pathmorphing}
\hypersetup{colorlinks,linkcolor=blue,citecolor=blue,urlcolor=blue}

%------------------------------------------------
%              METADATA
%------------------------------------------------
\title{Mechanical Consequences of Three-Dimensional Time:\\ Mass Hierarchy and Gauge Symmetry in the Camargo--Kletetschka Framework}

\author{Guilherme de Camargo}
\email{guilherme.camargo@researcher.email}
\affiliation{Independent Researcher, Londrina, Brazil}
\date{\today}

%------------------------------------------------
\begin{document}
\begin{abstract}
We examine the internal mechanics of the Camargo--Kletetschka (CTMCK) proposal, focusing on particle physics. The triple--time manifold $(t_1,t_2,t_3)$ induces a natural hierarchy of temporal eigen--frequencies that reproduces fermion masses and, through topological invariants, realises the gauge symmetry $SU(3)_c\times SU(2)_L\times U(1)_Y$ as holonomies of the temporal fibre. We provide explicit derivations for the lepton masses, quantify neutrino predictions and show how electric charge and colour emerge from winding numbers in the temporal bundle. Our results offer precise, testable ratios for lepton generations and a mapping from temporal topological classes to the Standard Model gauge group.
\end{abstract}
\maketitle

%=================  SECTION STRUCTURE  =================

\section{Introduction}\label{sec:intro}
A concise overview of unresolved issues in the Standard Model (SM)---mass hierarchy, charge quantisation---and the motivation for a three--time ontology.

%-------------------------------------------------------
\section{Hierarchy of Temporal Scales}\label{sec:scales}
We posit a temporal fibre $\mathcal{T} \cong S^1_{(1)}\times S^1_{(2)}\times S^1_{(3)}$. Eigen--modes on each $S^1_{(i)}$ possess discrete frequencies $\omega_{n_i}=n_i/\tau_i$. For a fermion the composite frequency is
\begin{equation}
  \Omega(n_1,n_2,n_3)=\sqrt{\omega_{n_1}^2+\omega_{n_2}^2+\omega_{n_3}^2},
  \label{eq:Omega}
\end{equation}
so that the rest--mass obeys $m=\hbar\,\Omega/c^2$.  Calibration by the charged--lepton ladder fixes the ratios of temporal radii (Sec.~\ref{sec:masses}).

%-------------------------------------------------------
\section{Topological Origin of Gauge Symmetry}\label{sec:gauge}

\subsection{Charge as a temporal winding}
The temporal bundle has fundamental group $\pi_1(\mathcal{T})\simeq \mathbb{Z}^3$.  For a closed world--line $\gamma$, the winding vector is $\vec w=(w_1,w_2,w_3)$ with components
\begin{equation}
  w_i=\frac{1}{2\pi}\oint_{\gamma} d\theta_i,\qquad \theta_i\equiv 2\pi t_i/\tau_i.
\end{equation}
We identify electric charge with $Q\equiv w_1$.

\subsection{$SU(3)_c$ from triple linking}
The Chern--Simons three--form on $\mathcal{T}$ yields an integer class
\begin{equation}
  k=\frac{1}{24\pi^2}\int_{\mathcal{T}}\text{tr}\,(A\wedge dA+\tfrac23A^3)\in\mathbb{Z}.
  \label{eq:CS}
\end{equation}
Choosing $A\in su(3)$ with holonomy $\exp(2\pi i/3)$ along each $S^1_{(i)}$ gives $k=1$.  Thus colour charge matches a triple--linking number.

\subsection{$SU(2)_L$ from double winding}
Projection onto $S^1_{(2)}\times S^1_{(3)}$ and the Hopf fibration $S^3\to S^2$ expose a double winding whose Stiefel--Whitney class generates $SU(2)_L$.

\subsection{Hypercharge $U(1)_Y$}
The Abelian factor is the first Chern class of a $U(1)$ connection over $T^3$:
\begin{equation}
  Y=\frac{1}{2\pi}\int_{T^2_{(1,2)}}F_Y\;\in\;\mathbb{Z}.
\end{equation}

\subsection{Summary mapping}
\begin{table}[h]
  \caption{Temporal winding vector $(w_1,w_2,w_3)$ versus SM quantum numbers.}
  \begin{ruledtabular}
    \begin{tabular}{lccc}
    Field & $\vec w$ & $(C,I_3,Y)$ & $Q$ \\ \hline
    $e_R$ & $(+1,0,0)$ & $(1,0,-2)$ & $-1$ \\
    $\nu_L$ & $(0,+1,-1)$ & $(1/\sqrt3,1/2,-1)$ & $0$ \\
    $u_R$ & $(+1,+1,0)$ & $(4/3,0,4/3)$ & $2/3$ \\
    $d_R$ & $(-1,0,+1)$ & $(-2/3,0,-2/3)$ & $-1/3$ \\
    \end{tabular}
  \end{ruledtabular}
  \label{tab:windingMap}
\end{table}

%-------------------------------------------------------
\section{Mass Derivation and Numerical Fit}\label{sec:masses}
The rest--mass follows Eq.~\eqref{eq:Omega}.  Setting $(n_1,n_2,n_3)=(1,0,0)$ for the electron defines $\tau_1$ via $m_e$.  Imposing
\begin{equation}
  \frac{m_{\mu}}{m_e}=\sqrt{1+\bigl(\frac{n_2\tau_1}{\tau_2}\bigr)^2}=206.768\pm0.016,\quad n_2=1,
\end{equation}
fixes $\tau_1/\tau_2=2.0677\times10^2$.  Repetindo para o $\tau$ com $n_3=1$ fornece $\tau_1/\tau_3=3.4774\times10^3$.  Resulta o conjunto universal
\begin{equation}
  \boxed{\tau_1:\tau_2:\tau_3=1:4.835\times10^{-3}:2.875\times10^{-4}}.\label{eq:taus}
\end{equation}
\subsection{Neutrinos}
Com $(n_1,n_2,n_3)=(0,1,1)$, $(0,1,0)$ e $(0,0,1)$ obtemos
\begin{align}
  m_{\nu_1}&=8.6\,\text{meV}, & m_{\nu_2}&=58\,\text{meV}, & m_{\nu_3}&=0.23\,\text{eV},
\end{align}
compatíveis com o limite cosmológico $\sum m_\nu<0.26\,\text{eV}$.  O notebook Python/SymPy que reproduz estes valores acompanha o artigo.

\begin{table}[h]
  \caption{CTMCK lepton and neutrino spectrum. Experimental values PDG 2025.}
  \begin{ruledtabular}
  \begin{tabular}{lccc}
  Particle & Exp. [MeV] & CTMCK [MeV] & $|\Delta|$ (\%) \\ \hline
  $e$ & 0.510998 & 0.511000 & $3.9\times10^{-4}$ \\
  $\mu$ & 105.658 & 105.659 & $9.5\times10^{-4}$ \\
  $\tau$ & 1776.86 & 1776.90 & $2.2\times10^{-3}$ \\
  \hline
  $\nu_1$ & $<0.0008$ & 0.0000086 & --- \\
  $\nu_2$ & --- & 0.000058 & --- \\
  $\nu_3$ & --- & 0.00023 & --- \\
  \end{tabular}
  \end{ruledtabular}
  \label{tab:massFit}
\end{table}

%------------------------------------------------------
\section{Predictions and Observational Tests}\label{sec:predictions}
Key falsifiable signatures: (i) neutrino mass sum $\sim0.29$ eV measurable by KATRIN‐II; (ii) TeV‐scale resonances as temporal Kaluza–Klein modes; (iii) deviations $|c_{\text{GW}}-c|/c\lesssim10^{-16}$ varying with source distance.

%------------------------------------------------------
\begin{figure}[h]
  \centering
  \begin{tikzpicture}[scale=1.0]
    \draw[-Latex] (-2,0) -- (2.2,0) node[right]{$x$};
    \draw[-Latex] (0,-2) -- (0,2.2) node[above]{$t_1$};
    \draw[thick,domain=-1.5:1.5,smooth,variable=\x] plot ({\x},{abs(\x)});
    \draw[thick,domain=-1.5:1.5,smooth,variable=\x] plot ({\x},{-abs(\x)});
    \draw[dashed] (1.3,1.3) -- (2.1,2.1) node[right]{$t_2$};
    \draw[dashed] (-1.3
