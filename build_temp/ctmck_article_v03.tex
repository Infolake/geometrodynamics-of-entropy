
%=================================================================
%  CTMCK – Observable Signatures of a Three‑Temporal Bounce‑Universe
%  Version 0.3 – JCAP pre‑submission (professional)
%=================================================================
\documentclass[reprint,amsmath,amssymb,aps,prd,nofootinbib,longbibliography]{revtex4-2}

% -------------------- PACKAGES --------------------
\usepackage[T1]{fontenc}
\usepackage{graphicx}
\graphicspath{{figures/}}      % Caminho das imagens
\usepackage{siunitx}
\usepackage{hyperref}
\usepackage{physics}
\usepackage{mathtools}
\usepackage{color}

% Definir cor personalizada
\definecolor{RoyalBlue}{RGB}{65,105,225}

% hyperlink setup
\hypersetup{
    colorlinks=true,
    linkcolor=black,        % referências internas em preto
    citecolor=RoyalBlue,    % citações em azul discreto
    urlcolor=RoyalBlue,     % URLs e e-mails em azul discreto
    pdftitle={Observable Signatures of a Three-Temporal Bounce‑Universe Inside a Black Hole},
    pdfkeywords={multidimensional time, Einstein–Cartan bounce, cosmological tensions, JWST anomalies, neutrino mass}
}

\preprint{CTMCK‑v0.3}

\begin{document}

\title{Observable Signatures of a Three‑Temporal Bounce‑Universe Inside a Black Hole}

\author{Guilherme de Camargo}
\email{guilherme@medsuite.com.br}
\affiliation{Independent Researcher, Londrina, PR 86010‑260, Brazil}
\date{\today}

\keywords{multidimensional time, Einstein–Cartan bounce, cosmological tensions, JWST anomalies, neutrino mass}
\pacs{04.20.Cv, 98.80.Qc, 04.50.Kd, 14.60.Pq}

% -------------------- ABSTRACT --------------------
\begin{abstract}
We extend Kletetschka's three‑dimensional‑time framework to a bouncing black‑hole cosmology and show that the resulting \textit{Camargo–Kletetschka Multidimensional Temporal Cosmogenesis} (CTMCK) scenario simultaneously (i) alleviates the persistent $H_0$ and $S_8$ tensions, (ii) accounts for the over‑abundance of ultra‑massive $z>10$ galaxies reported by \textit{JWST}, and (iii) predicts a neutrino‑mass sum $\Sigma m_\nu = 0.29\pm0.05\,$eV compatible with the most recent HiLLiPoP\,+\,DESI bound $\Sigma m_\nu<0.30\,$eV (95\% C.L.). \\[2pt]
Key falsifiable signatures include a stochastic gravitational‑wave background peaking at $f_b\simeq\SI{100}{\micro\hertz}$—optimally placed for \textsc{LISA}—and a distinctive evolution of the effective dark‑energy equation‑of‑state $w(z)=-1+0.05(1+z)^3$. We present full derivations of (i) the temporal‑radius hierarchy $\tau_1\!:\!\tau_2\!:\!\tau_3 = 1\!:\!4.835\times10^{-3}\!:\!2.875\times10^{-4}$ from charged‑lepton masses, (ii) the Einstein–Cartan bounce density $\rho_{\text{bounce}}=(c^7/G^2\hbar)\,\tau_1^{-1}\tau_2^{-1}\tau_3^{-1}$ and (iii) the gravitational‑wave spectrum $\Omega_{\mathrm{GW}}(f)$. A near‑term experimental roadmap (2025‑2032) is outlined.
\end{abstract}

\maketitle

% ================== MAIN TEXT =====================

\input{sections/introduction.tex}
\input{sections/framework.tex}
\input{sections/bounce.tex}
\input{sections/tensions.tex}
\input{sections/predictions.tex}
\input{sections/roadmap.tex}
\input{sections/discussion.tex}
\input{sections/conclusions.tex}

% -------------------- ACKS ------------------------
\begin{acknowledgments}
I thank G.~Kletetschka for foundational discussions on three‑dimensional time theory. I am also grateful to the \textit{JWST} collaboration for providing the observational impetus for this work.
\end{acknowledgments}

% -------------------- BIBLIOGRAPHY ----------------
\bibliographystyle{apsrev4-2}
\bibliography{references}

\end{document}
