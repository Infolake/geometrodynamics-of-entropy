\documentclass[reprint,aps,prd,amsmath,amssymb,
               nofootinbib,longbibliography]{revtex4-2}

% -------------------- PACKAGES --------------------
\usepackage[T1]{fontenc}
\usepackage{graphicx}
  \graphicspath{{figures/}}   % <— PNG directory
\usepackage{siunitx}
\usepackage{physics}
\usepackage{mathtools}
\usepackage{xcolor}
\usepackage{hyperref}

% --- colours & links ---
\definecolor{RoyalBlue}{RGB}{65,105,225}
\hypersetup{
  colorlinks = true,
  linkcolor  = black,
  citecolor  = RoyalBlue,
  urlcolor   = RoyalBlue,
  pdftitle   = {Observable Signatures of a Three-Temporal Bounce-Universe Inside a Black Hole},
  pdfkeywords= {multidimensional time, Einstein–Cartan bounce, cosmological tensions,
                JWST anomalies, neutrino mass}
}
\preprint{CTMCK-v0.3}

% -------------------- FRONT MATTER ----------------
\begin{document}

\title{Observable Signatures of a Three-Temporal Bounce-Universe Inside a Black Hole}

\author{Guilherme de Camargo}
\email{guilherme@medsuite.com.br}
\affiliation{Independent Researcher, Londrina–PR 86010-260, Brazil}
\date{\today}

\keywords{multidimensional time, Einstein–Cartan bounce, cosmological tensions,
          JWST anomalies, neutrino mass}
\pacs{04.20.Cv, 98.80.Qc, 04.50.Kd, 14.60.Pq}

% -------------------- ABSTRACT --------------------
\begin{abstract}
We extend Kletetschka's three-dimensional-time framework to a bouncing black-hole
cosmology and show that the resulting
\textit{Camargo–Kletetschka Multidimensional Temporal Cosmogenesis} (CTMCK)
scenario simultaneously (i) alleviates the persistent $H_0$ and $S_8$ tensions,
(ii) accounts for the over-abundance of ultra-massive $z>10$ galaxies reported by
\textit{JWST}, and (iii) predicts a neutrino-mass sum
$\Sigma m_\nu = 0.29\pm0.05\,\text{eV}$ compatible with the most recent
HiLLiPoP+DESI bound $\Sigma m_\nu < 0.30\,\text{eV}$ (95\% C.L.).

Key falsifiable signatures include a stochastic gravitational-wave background
peaking at $f_b \simeq \SI{100}{\micro\hertz}$—optimally placed for
\textsc{LISA}—and a distinctive evolution of the effective dark-energy
equation-of-state $w(z) = -1 + 0.05(1+z)^3$.
We present full derivations of:
(1) the temporal-radius hierarchy
$\tau_1\!:\!\tau_2\!:\!\tau_3 = 1\!:\!4.835\times10^{-3}\!:\!2.875\times10^{-4}$
from charged-lepton masses;
(2) the Einstein–Cartan bounce density
$\rho_{\mathrm{bounce}} = (c^7/G^2\hbar)\,\tau_1^{-1}\tau_2^{-1}\tau_3^{-1}$;
and (3) the gravitational-wave spectrum $\Omega_{\mathrm{GW}}(f)$.
A near-term experimental roadmap (2025–2032) is outlined.
\end{abstract}

\maketitle

% ==================================================
\section{Introduction}\label{sec:intro}

\subsection{The crisis in standard cosmology}
Precision data expose two $\gtrsim\!4\sigma$ inconsistencies inside
$\Lambda$CDM: the Hubble-parameter tension
($H_0^{\mathrm{local}} = 74.0\pm1.0$ versus
$H_0^{\mathrm{CMB}} = 67.4\pm0.5$\,km\,s$^{-1}$\,Mpc$^{-1}$) and the structure-growth
tension ($S_8^{\mathrm{WL}}\!\simeq\!0.75$ vs.\ $0.83$).
\textit{JWST} adds a third anomaly: massive
($\gtrsim10^{10}\,M_\odot$) galaxies already at $z>10$.

\subsection{Overview of CTMCK}
CTMCK posits a six-dimensional manifold
$\mathcal{M}^6 = \mathbb{R}^3 \times T^3$,
with three compact temporal circles $T^3$. An Einstein–Cartan
torsion term halts collapse of a parent-universe black hole,
producing a nonsingular bounce whose interior is our cosmos. The evolving
temporal radii $\tau_i(t)$ modify Friedmann dynamics,
shift $(H_0,S_8)$, and generate a LISA-band
stochastic gravitational-wave background.

% ==================================================
\section{Three-dimensional time framework}\label{sec:framework}

\subsection{6-D metric and stability}
\begin{equation}
ds^2 = -c^2\!\bigl(dt_1^2 + \alpha\,dt_2^2 + \beta\,dt_3^2\bigr)
       + dx^2 + dy^2 + dz^2 ,
\end{equation}
$\alpha=(\tau_2/\tau_1)^2,\; \beta=(\tau_3/\tau_1)^2$.
Appendix D proves hyperbolicity for the observed hierarchy.

\subsection{Temporal-radius hierarchy}
Solving the charged-lepton system (App.\,A) yields
\begin{equation}
\tau_1 : \tau_2 : \tau_3 \;=\; 1 : 4.835\times10^{-3} : 2.875\times10^{-4}.
\end{equation}

% ==================================================
\section{Einstein–Cartan bounce cosmology}\label{sec:bounce}
Fermion spin sources torsion $S^\lambda_{\mu\nu}$.
Collapse halts when
\begin{equation}
\rho_{\mathrm{bounce}}
  = \frac{c^7}{G^2\hbar}\,\tau_1^{-1}\tau_2^{-1}\tau_3^{-1}.
\end{equation}
The bounce re-expands, carrying pre-bounce information that seeds early
structure formation.

% ==================================================
\section{Resolution of cosmological tensions}\label{sec:tensions}
The term $\sum_i \ddot{\tau}_i/\tau_i$ acts as effective dark
energy, yielding $H_0=71.2\pm1.8$\,km\,s$^{-1}$\,Mpc$^{-1}$ and
$S_8=0.783\pm0.024$ (Fig.\,\ref{fig:tensions}).

% ==================================================
\section{Testable predictions}\label{sec:predictions}

\paragraph*{Neutrino masses.}
Temporal eigenmodes give
$(m_{\nu_1},m_{\nu_2},m_{\nu_3})=(8.6,58,230)$\,meV
$\Rightarrow \Sigma m_\nu = 0.29\pm0.05$\,eV.

\paragraph*{GW background.}
\begin{equation}
\Omega_{\mathrm{GW}}(f)
   = A\!\left(\frac{f}{1\,\mu\text{Hz}}\right)^{-2/3}
     \exp\!\Bigl[-\frac{f}{f_b}\Bigr], \quad f_b\simeq100\,\mu\text{Hz},
\end{equation}
within \textsc{LISA}'s peak sensitivity (Fig.\,\ref{fig:gw}).

% ==================================================
\section{Experimental roadmap}\label{sec:roadmap}
\textbf{2025–27:} KATRIN-II, DESI-Y5, JWST Cycle 3.\\
\textbf{2028–32:} \textsc{LISA}, Euclid + Roman, CMB-S4.\\
\textbf{>2032:} Einstein Telescope, UHE-cosmic-ray KK searches
(Fig.\,\ref{fig:timeline}).

% ==================================================
\section{Discussion}\label{sec:discussion}
CTMCK ties quantum mechanics, general relativity and cosmology
through temporal geometry, offering a single geometric driver
for multiple anomalies and explicit falsifiability.

% ==================================================
\section{Conclusions}\label{sec:conclusions}
Compact temporal dimensions inside an Einstein–Cartan bounce
provide a coherent alternative to $\Lambda$CDM, with critical tests
imminent via neutrino physics, GW astronomy and deep-field surveys.

% -------------------- ACKNOWLEDGMENTS -------------
\begin{acknowledgments}
I thank G.~Kletetschka for foundational discussions on three-dimensional
time theory and the \textit{JWST} collaboration for the observational impetus.
\end{acknowledgments}

% -------------------- FIGURES ---------------------
\begin{figure}[t]
  \centering\includegraphics[width=\linewidth]{fig1_geometry_6d}
  \caption{Six-dimensional CTMCK manifold: three temporal circles
           ($t_1,t_2,t_3$) and emergent 3-space.}
  \label{fig:geometry}
\end{figure}

\begin{figure}[t]
  \centering\includegraphics[width=\linewidth]{fig3_tensions_resolution}
  \caption{CTMCK intermediate values resolve both
           the $H_0$ (left) and $S_8$ (right) tensions.}
  \label{fig:tensions}
\end{figure}

\begin{figure}[t]
  \centering\includegraphics[width=\linewidth]{fig2_gw_spectrum}
  \caption{Predicted stochastic GW background—peak at
           $f_b\simeq\SI{100}{\micro\hertz}$ lies in the \textsc{LISA} band.}
  \label{fig:gw}
\end{figure}

\begin{figure}[t]
  \centering\includegraphics[width=\linewidth]{fig5_predictions_timeline}
  \caption{Timeline of experimental tests for CTMCK predictions
           (2025–2032).}
  \label{fig:timeline}
\end{figure}

% -------------------- BIBLIOGRAPHY ---------------
\bibliographystyle{apsrev4-2}
\bibliography{references}

\end{document}
```

### Como usar

1. Salve o bloco acima em `ctmck_article_v03.tex`.
2. Mantenha apenas os PNG estáticos indicados.
3. Compile com:

   ```bash
   pdflatex ctmck_article_v03
   bibtex   ctmck_article_v03
   pdflatex ctmck_article_v03
   pdflatex ctmck_article_v03
   ```
4. Empacote para arXiv junto com `references.bib`, `figures/…`, e (opcional) `README_ARXIV.txt`.

Pronto!
