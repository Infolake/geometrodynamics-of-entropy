% ========= CTMCK Letter v0.1b =========
\documentclass[reprint,amsmath,amssymb,aps,prd,nofootinbib]{revtex4-2}

% ------ Packages ------
\usepackage[T1]{fontenc}
\usepackage{graphicx}
\usepackage{hyperref}
\usepackage{physics}
\usepackage{times}

% ------ Metadata ------
\hypersetup{
  pdftitle={Observable Signatures of a Three–Temporal Bounce‐Universe Inside a Black Hole},
  pdfauthor={Guilherme de Camargo},
  pdfkeywords={multidimensional time, Einstein–Cartan bounce, black‐hole cosmology, JWST anomalies, neutrino mass}
}

% ------ Document ------
\begin{document}

% -------- Title block --------
\title{Observable Signatures of a Three‑Temporal Bounce‑Universe Inside a Black Hole}

\author{Guilherme de Camargo}
\email{guilherme@medsuite.com.br}
\affiliation{Independent Researcher, Londrina–PR, Brazil\\ORCID: 0009-0004-8913-9419}

\date{\today}
\begin{abstract}
We extend the recent three‑dimensional‑time framework of Kletetschka to the cosmological regime and propose that the observable Universe is the interior of a spinning black hole formed in a parent cosmos.  The resulting \emph{Cosmogênese Temporal Multidimensional Camargo–Kletetschka} (CTMCK) scenario replaces the initial singularity by an Einstein–Cartan bounce, yields a global Schwarzschild parameter $2GM/(c^{2}R)\approx1$, and naturally explains (i) the abundance of ultra‑massive galaxies at redshifts $z\!>\!10$ observed by the \emph{James Webb Space Telescope} and (ii) the handedness bias in galaxy rotation.  In the particle sector CTMCK reproduces charged‑lepton masses and predicts a definite neutrino mass sum $\sum m_{\nu}=0.29\,\mathrm{eV}$.  A concise set of falsifiable signatures—LISA‑band torsional waves and TeV Kaluza–Klein resonances—places the theory within reach of forthcoming experiments.
\end{abstract}

\maketitle

% -------- 1. Introduction --------
\section{Motivation}
A full synthesis of quantum theory and gravity should explain both micro‑scales (particle masses, charges) and macro‑scales (cosmic initial conditions, dark energy) within a single geometric principle.  Kletetschka's three‑temporal formalism\,\cite{Kletetschka2025}, in which time carries three orthogonal coordinates $(t_1,t_2,t_3)$ and space emerges secondarily, offers precisely such a unifying seed.  We show that minimal extensions of this idea resolve key observational puzzles while remaining sharply testable.

% -------- 2. Foundational premises --------
\section{Foundational premises of three‑temporal physics}
\textbf{(P1) Metric:} a six‑dimensional manifold $\mathcal M^6$ with line element
\begin{equation}\label{eq:metric}
  ds^{2}=dt_{1}^{2}+dt_{2}^{2}+dt_{3}^{2}-dx^{2}-dy^{2}-dz^{2}.
\end{equation}
\textbf{(P2) Mass hierarchy:} modes on compact circles $S_{(i)}^{1}$ of radii $\tau_i$ produce lepton and quark masses via $m=\hbar c^{-2}\sqrt{\sum\omega_{n_i}^{2}}$.  Using electron, muon and tau fixes $\tau$ (no Yukawas).

\textbf{(P3) Topological charge:} a winding vector $\vec{w}\in\mathbb Z^{3}$ on the temporal torus generates the $U(1)\times SU(2)\times SU(3)$ charges.

\textbf{(P4) Einstein–Cartan torsion:} spin–density $\tau^{\lambda}_{\;\mu\nu}$ enters the connection, becoming repulsive at Planck density and realising a nonsingular bounce.

\textbf{Original extension (this Letter)}: the bounce‑interior matches a parent black hole, providing inherited overdensities and a global spin—two parameters that map onto JWST anomalies without additional fields.

% -------- 3. Minimal cosmological dynamics --------
\section{Bounce inside a black hole}
Including torsion yields the modified Friedmann equation
\begin{equation}
  \Bigl(\dot a/a\Bigr)^{2}=\frac{8\pi G}{3}\rho-\frac{k}{a^{2}}+\frac{\kappa^{2}}{24}\sigma^{2}, \quad \sigma^{2}=\tau^{ABC}\tau_{ABC}.
\end{equation}
The Universe attains a minimum radius $a_{\rm min}$ when $\sigma^{2}=\sigma_{\rm crit}^{2}$ and rebounds.  Matching $M_{U}$ and $R_{\text{obs}}$ gives $2GM/(c^{2}R)\simeq0.98$—consistent with an interior‑BH interpretation.

% -------- 4. Observable signatures --------
\section{Observable signatures}
\begin{enumerate}
  \item \textbf{JWST ultra‑massive galaxies}: inherited seeds shorten collapse times; CTMCK fits the Boylan‑Kolchin tension.
  \item \textbf{Preferred spin axis}: parent BH angular momentum predicts the Shamir rotation bias.
  \item \textbf{Neutrino mass sum}: $\sum m_{\nu}=0.29\,\mathrm eV$ (KATRIN II sensitivity).
  \item \textbf{LISA torsional bump}: peak strain at $f\!\simeq\!10^{-2}\,$Hz.
  \item \textbf{TeV KK resonances}: narrow states at $2.3$ and $4.1$ TeV (future collider reach).
\end{enumerate}

% -------- 5. Outlook --------
\section{Outlook}
CTMCK condenses four outstanding problems—hierarchy, singularity, early structures, dark sector—into a single geometric hypothesis.  Its decisive predictions will be probed within a decade, rendering the framework imminently falsifiable.

% -------- Acknowledgments --------
% -------- Acknowledgments --------
\begin{acknowledgments}
No external funding was received.
\end{acknowledgments}

% -------- References --------
\begin{thebibliography}{99}
\bibitem{Kletetschka2025}G.~Kletetschka, Rep.~Adv.~Phys.~Sci. \textbf{9}, 2550004 (2025).
\bibitem{BoylanKolchin2023}M.~Boylan‑Kolchin, Nat. Astron. \textbf{7}, 147 (2023).
\bibitem{Shamir2025}L.~Shamir, Mon. Not. R. Astron. Soc. \textbf{520}, 4607 (2025).
\end{thebibliography}
\end{document} 