%=================================================================
%  CTMCK – Observable Signatures of a Three-Temporal Bounce-Universe
%  Monolithic version 0.3  (sections expanded inline)
%=================================================================
\documentclass[reprint,amsmath,amssymb,aps,prd,nofootinbib,longbibliography]{revtex4-2}

% -------------------- PACKAGES --------------------
\usepackage[T1]{fontenc}
\usepackage{graphicx}
\graphicspath{{figures/}}
\usepackage{siunitx}
\usepackage{hyperref}
\usepackage{physics}
\usepackage{mathtools}
\usepackage{color}
\definecolor{RoyalBlue}{RGB}{65,105,225}
\hypersetup{
    colorlinks   = true,
    linkcolor    = black,
    citecolor    = RoyalBlue,
    urlcolor     = RoyalBlue,
    pdftitle     = {Observable Signatures of a Three-Temporal Bounce-Universe Inside a Black Hole},
    pdfkeywords  = {multidimensional time, Einstein–Cartan bounce, cosmological tensions, JWST anomalies, neutrino mass}
}
\preprint{CTMCK-v0.3}

% -------------------- FRONT-MATTER ----------------
\begin{document}
\title{Observable Signatures of a Three-Temporal Bounce-Universe Inside a Black Hole}
\author{Guilherme de Camargo}
\email{guilherme@medsuite.com.br}
\affiliation{Independent Researcher, Londrina, PR 86010-260, Brazil}
\date{\today}
\keywords{multidimensional time, Einstein–Cartan bounce, cosmological tensions, JWST anomalies, neutrino mass}
\pacs{04.20.Cv, 98.80.Qc, 04.50.Kd, 14.60.Pq}

% -------------------- ABSTRACT --------------------
\begin{abstract}
We extend Kletetschka's three-dimensional-time framework to a bouncing black-hole cosmology and show that the resulting \textit{Camargo–Kletetschka Multidimensional Temporal Cosmogenesis} (CTMCK) scenario simultaneously (i) alleviates the persistent $H_0$ and $S_8$ tensions, (ii) accounts for the over-abundance of ultra-massive $z>10$ galaxies reported by \textit{JWST}, and (iii) predicts a neutrino-mass sum $\Sigma m_\nu = 0.29\pm0.05\,$eV compatible with the most recent HiLLiPoP\,+\,DESI bound $\Sigma m_\nu<0.30\,$eV (95\% C.L.). \\[2pt]
The \textit{Hubble tension} represents a $4\sigma$ discrepancy between local distance-ladder measurements ($H_0 = 74.0\pm1.0\,$km\,s$^{-1}$\,Mpc$^{-1}$) and the \textit{Planck} CMB inference ($H_0 = 67.4\pm0.5$), and  
(ii) the \textit{structure-growth tension} parameterised by $S_8$, where weak-lensing surveys favour $S_8\simeq0.75$ while the CMB yields $S_8\simeq0.83$.  
The discovery of massive $(\gtrsim10^{10}\,M_\odot)$ galaxies at $z\gtrsim10$ by \textit{JWST} adds a third anomaly.

\subsection{Overview of the CTMCK Framework}
CTMCK postulates a six-dimensional manifold $\mathcal{M}^6=\mathbb{R}^3\!\times\!T^3$ with three compact temporal circles $T^3=S^1_{(1)}\times S^1_{(2)}\times S^1_{(3)}$ of radii $\tau_i$. Gravitational collapse in a parent universe triggers an Einstein–Cartan \emph{bounce} that seeds our Universe inside the would-be black-hole interior. Temporal dynamics modify the Friedmann equations, naturally shift $H_0$ and $S_8$, and generate a LISA-band stochastic gravitational-wave background.
\end{abstract}

\maketitle

% ==================================================
% 2. 3D-TIME FRAMEWORK
% ==================================================
\section{Three-Dimensional Time Framework}\label{sec:framework}
\subsection{6D Metric and Stability}
We adopt the line element
\begin{equation}
ds^2=-c^2\!\left(dt_1^2+\alpha\,dt_2^2+\beta\,dt_3^2\right)+dx^2+dy^2+dz^2 ,
\end{equation}
where $\alpha=(\tau_2/\tau_1)^2$ and $\beta=(\tau_3/\tau_1)^2$.  
Appendix~D demonstrates hyperbolic stability ($\alpha,\beta>0$) and the absence of closed timelike curves for the observed hierarchy.

\subsection{Temporal-Radius Hierarchy}
Solving the charged-lepton system (Appendix A) yields
\begin{equation}
\tau_1 : \tau_2 : \tau_3 = 1 : 4.835\times10^{-3} : 2.875\times10^{-4}\; .
\end{equation}

% ==================================================
% 3. BOUNCE COSMOLOGY
% ==================================================
\section{Einstein–Cartan Bounce Cosmology}\label{sec:bounce}
In Einstein–Cartan gravity the intrinsic fermion spin sources torsion, producing a repulsive term that halts collapse at density
\begin{equation}
\rho_{\text{bounce}}=\frac{c^7}{G^2\hbar}\,\tau_1^{-1}\tau_2^{-1}\tau_3^{-1}.
\end{equation}
The interior re-expands, inheriting pre-bounce information and seeding rapid structure formation.

% ==================================================
% 4. RESOLUTION OF TENSIONS
% ==================================================
\section{Resolution of Cosmological Tensions}\label{sec:tensions}
Temporal-scale evolution introduces a positive acceleration term $\sum_i\ddot{\tau_i}/\tau_i$ in the Friedmann equation, raising $H_0$ to $71.2\pm1.8$ km s$^{-1}$ Mpc$^{-1}$ and reducing $S_8$ to $0.783\pm0.024$ (Fig.\,\ref{fig:tensions}).

% ==================================================
% 5. TESTABLE PREDICTIONS
% ==================================================
\section{Testable Predictions}\label{sec:predictions}
\subsection{Neutrino Masses}
Temporal eigenmodes give $(m_{\nu_1},m_{\nu_2},m_{\nu_3})=(8.6,58,230)$ meV, hence $\Sigma m_\nu=0.29\pm0.05$ eV.
\subsection{Gravitational-Wave Background}
Bounce fluctuations generate
\begin{equation}
\Omega_{\mathrm{GW}}(f)=A\left(\frac{f}{1\ \mu{\rm Hz}}\right)^{-2/3}\!\!\exp\!\left(-\frac{f}{f_b}\right),\qquad f_b\!\simeq\!100\;\mu{\rm Hz},
\end{equation}
detectable by \textsc{LISA} (Fig.\,\ref{fig:gw}).

% ==================================================
% 6. EXPERIMENTAL ROADMAP
% ==================================================
\section{Experimental Roadmap}\label{sec:roadmap}
2025–27: KATRIN-II (neutrino mass), DESI-Y5 ($S_8$), JWST Cycle 3 ($z>15$ galaxies).  
2028–32: \textsc{LISA} (GW background), Euclid + Roman (dark-energy evolution), CMB-S4 (cosmological parameters).  
Post-2032: Einstein Telescope, ultra-high-energy cosmic-ray KK searches.

% ==================================================
% 7. DISCUSSION
% ==================================================
\section{Discussion}\label{sec:discussion}
CTMCK unifies quantum mechanics and general relativity via temporal geometry, addresses three independent cosmological anomalies, and offers multiple near-term falsifiable predictions.

% ==================================================
% 8. CONCLUSIONS
% ==================================================
\section{Conclusions}\label{sec:conclusions}
The CTMCK framework provides a coherent, testable alternative to $\Lambda$CDM by attributing cosmic acceleration and particle mass hierarchies to compact temporal dimensions inside a black-hole bounce. Upcoming measurements by KATRIN-II, \textsc{LISA} and JWST will critically probe the theory.

% ==================================================
% ACKNOWLEDGMENTS
% ==================================================
\begin{acknowledgments}
I thank G.~Kletetschka for foundational discussions on three-dimensional time theory. I am also grateful to the \textit{JWST} collaboration for providing the observational impetus for this work.
\end{acknowledgments}

% -------------------- FIGURES ---------------------
\begin{figure}[t]
  \centering
  \includegraphics[width=\linewidth]{fig1_geometry_6d}
  \caption{Six-dimensional CTMCK manifold showing three temporal circles and emergent 3-space.}
  \label{fig:geometry}
\end{figure}

\begin{figure}[t]
  \centering
  \includegraphics[width=\linewidth]{fig3_tensions_resolution}
  \caption{CTMCK intermediate values resolve the $H_0$ and $S_8$ tensions.}
  \label{fig:tensions}
\end{figure}

\begin{figure}[t]
  \centering
  \includegraphics[width=\linewidth]{fig4_gw_spectrum}
  \caption{Predicted stochastic GW background; peak at $f_b\simeq100\ \mu$Hz lies in the \textsc{LISA} band.}
  \label{fig:gw}
\end{figure}

% -------------------- BIBLIOGRAPHY ---------------
\bibliographystyle{apsrev4-2}
\bibliography{references}
\end{document}